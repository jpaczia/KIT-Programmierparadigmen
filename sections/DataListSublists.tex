\section{Sublists}
\subsection{Extracting sublists}

\funcSignature{take::Int\rightarrow [a]\rightarrow [a] }
take n, applied to a list xs, returns the prefix of xs of length n, or xs itself if n \textgreater length xs:
\begin{align*}
	&\text{take 5 \enquote{Hello World!} == \enquote{Hello}}\\
	&\text{take 3 [1,2,3,4,5] == [1,2,3]}\\
	&\text{take 3 [1,2] == [1,2]}\\
	&\text{take 3 [] == []}\\
	&\text{take (-1) [1,2] == []}\\
	&\text{take 0 [1,2] == []}
\end{align*}
It is an instance of the more general genericTake, in which n may be of any integral type.

\funcSignature{drop::Int\rightarrow [a]\rightarrow [a] }
drop n xs returns the suffix of xs after the first n elements, or [] if n \textgreater length xs:
\begin{align*}
	&\text{drop 6 \enquote{Hello World!} == \enquote{World!}}\\
	&\text{drop 3 [1,2,3,4,5] == [4,5]}\\
	&\text{drop 3 [1,2] == []}\\
	&\text{drop 3 [] == []}\\
	&\text{drop (-1) [1,2] == [1,2]}\\
	&\text{drop 0 [1,2] == [1,2]}\\
\end{align*}
It is an instance of the more general genericDrop, in which n may be of any integral type.

\funcSignature{splitAt::Int\rightarrow [a]\rightarrow ([a], [a])}
splitAt n xs returns a tuple where first element is xs prefix of length n and second element is the remainder of the list:
\begin{align*}
	&\text{splitAt 6 \enquote{Hello World!} == (\enquote{Hello }, \enquote{World!})}\\
	&\text{splitAt 3 [1,2,3,4,5] == ([1,2,3],[4,5])}\\
	&\text{splitAt 1 [1,2,3] == ([1],[2,3])}\\
	&\text{splitAt 3 [1,2,3] == ([1,2,3],[])}\\
	&\text{splitAt 4 [1,2,3] == ([1,2,3],[])}\\
	&\text{splitAt 0 [1,2,3] == ([],[1,2,3])}\\
	&\text{splitAt (-1) [1,2,3] == ([],[1,2,3])}
\end{align*}
It is equivalent to (take n xs, drop n xs) when n is not \_\textbar \_ (splitAt \_\textbar\_ xs = \_\textbar\_). splitAt is an instance of the more general genericSplitAt, in which n may be of any integral type.

\funcSignature{takeWhile::(a\rightarrow Bool)\rightarrow [a]\rightarrow [a] }
takeWhile, applied to a predicate p and a list xs, returns the longest prefix (possibly empty) of xs of elements that satisfy p:
\begin{align*}
	&takeWhile\; (< 3)\; [1,2,3,4,1,2,3,4] == [1,2]\\
	&takeWhile\; (< 9)\; [1,2,3] == [1,2,3]\\
	&takeWhile\; (< 0)\; [1,2,3] == []
\end{align*}

\funcSignature{dropWhile::(a\rightarrow Bool)\rightarrow [a]\rightarrow [a] }
dropWhile p xs returns the suffix remaining after takeWhile p xs:
\begin{align*}
	&dropWhile\; (< 3)\; [1,2,3,4,5,1,2,3] == [3,4,5,1,2,3]\\
	&dropWhile\; (< 9)\; [1,2,3] == []\\
	&dropWhile\; (< 0)\; [1,2,3] == [1,2,3]
\end{align*}

\funcSignature{dropWhileEnd::(a\rightarrow Bool)\rightarrow [a]\rightarrow [a] }
The dropWhileEnd function drops the largest suffix of a list in which the given predicate holds for all elements. For example:
\begin{align*}
	&\text{dropWhileEnd isSpace \enquote{"foo\textbackslash n"} == \enquote{foo}}\\
	&\text{dropWhileEnd isSpace \enquote{foo bar} == \enquote{foo bar}}\\
	&\text{dropWhileEnd isSpace (\enquote{foo\textbackslash n} ++ undefined) == \enquote{foo} ++ undefined}
\end{align*}
Since: 4.5.0.0

\funcSignature{span::(a\rightarrow Bool)\rightarrow [a]\rightarrow ([a], [a]) }
span, applied to a predicate p and a list xs, returns a tuple where first element is longest prefix (possibly empty) of xs of elements that satisfy p and second element is the remainder of the list:
\begin{align*}
	&span\; (< 3)\; [1,2,3,4,1,2,3,4] == ([1,2],[3,4,1,2,3,4])\\
	&span\; (< 9)\; [1,2,3] == ([1,2,3],[])\\
	&span\; (< 0)\; [1,2,3] == ([],[1,2,3])
\end{align*}
span p xs is equivalent to (takeWhile p xs, dropWhile p xs)

\funcSignature{break::(a\rightarrow Bool)\rightarrow [a]\rightarrow ([a], [a]) }
break, applied to a predicate p and a list xs, returns a tuple where first element is longest prefix (possibly empty) of xs of elements that do not satisfy p and second element is the remainder of the list:
\begin{align*}
	&\text{break (\textgreater 3) [1,2,3,4,1,2,3,4] == ([1,2,3],[4,1,2,3,4])}\\
	&\text{break (\textless 9) [1,2,3] == ([],[1,2,3])}\\
	&\text{break (\textgreater 9) [1,2,3] == ([1,2,3],[])}\\
	&\text{break even [1,2,3] = ([1], [2,3])}
\end{align*}
break p is equivalent to span (not . p).

\eqCode
\begin{lstlisting}[frame=single]
splitWhen pred [] = ([], [])
splitWhen pred (x:xs)
	|pred x = ([], (x:xs))
	|otherwise = let (ys, zs) = splitWhen pred xs in (x:ys, zs)
\end{lstlisting}

\funcSignature{stripPrefix::Eq\; a\Rightarrow [a]\rightarrow [a]\rightarrow Maybe\; [a] }
The stripPrefix function drops the given prefix from a list. It returns Nothing if the list did not start with the prefix given, or Just the list after the prefix, if it does.
\begin{align*}
	&\text{stripPrefix \enquote{foo} \enquote{foobar} == Just \enquote{bar}}\\
	&\text{stripPrefix \enquote{foo} \enquote{foo} == Just ""}\\
	&\text{stripPrefix \enquote{foo} \enquote{barfoo} == Nothing}\\
	&\text{stripPrefix \enquote{foo} \enquote{barfoobaz} == Nothing}
\end{align*}

\checkImpl
\begin{lstlisting}[frame=single]
stripPrefix::Eq a => [a] -> [a] -> Maybe [a]
stripPrefix [] ys = Just ys
stripPrefix (x:xs) (y:ys)
  | x == y = stripPrefix xs ys
stripPrefix _ _ = Nothing
\end{lstlisting}

\funcSignature{group::Eq\; a\Rightarrow [a]\rightarrow [[a]] }
The group function takes a list and returns a list of lists such that the concatenation of the result is equal to the argument. Moreover, each sublist in the result contains only equal elements. For example,
\codeEx{group \enquote{Mississippi} = [\enquote{M},\enquote{i},\enquote{ss},\enquote{i},\enquote{ss},\enquote{i},\enquote{pp},\enquote{i}]}
It is a special case of groupBy, which allows the programmer to supply their own equality test.

\eqCode
\label{haskellBreak}
\begin{lstlisting}[frame=single]
groups:: Eq t => [t] -> [[t]]
groups [] = []
groups (x:xs) = let (h, r) = break (/=x) xs in (x:h):groups r
\end{lstlisting}

\funcSignature{inits::[a]\rightarrow [[a]] }
The inits function returns all initial segments of the argument, shortest first. For example,
\codeEx{inits "abc" == ["","a","ab","abc"]}
Note that inits has the following strictness property: inits (xs ++ \_\textbar\_) = inits xs ++ \_\textbar\_ \\
In particular, inits \_\textbar\_ = [] : \_\textbar\_

\begin{lstlisting}[frame=single]
inits:: [a] -> [[a]]
inits [] = [[]]
inits (x:xs) = [[]] ++ map (x:) (inits xs)
\end{lstlisting}

\funcSignature{tails::[a]\rightarrow [[a]] }
The tails function returns all final segments of the argument, longest first. For example,
\codeEx{tails \enquote{abc} == [\enquote{abc}, \enquote{bc}, \enquote{c}, \enquote{}]}
Note that tails has the following strictness property: tails \_\textbar\_ = \_\textbar\_ : \_\textbar\_
\begin{lstlisting}[frame=single]
tails::[a] -> [[a]]
tails [] =  [[]]
tails xxs@(_:xs) = xxs : tails xs
\end{lstlisting}

\subsection{Predicates}
\funcSignature{isPrefixOf::Eq\; a\Rightarrow [a]\rightarrow [a]\rightarrow Bool}
The isPrefixOf function takes two lists and returns True iff the first list is a prefix of the second.
\begin{lstlisting}[frame=single]
isPrefixOf::(Eq a) => [a] -> [a] -> Bool
isPrefixOf [] _ = True
isPrefixOf _  [] = False
isPrefixOf (x:xs) (y:ys)= x == y && isPrefixOf xs ys
\end{lstlisting}

\funcSignature{isSuffixOf::Eq\; a\Rightarrow [a]\rightarrow [a]\rightarrow Bool}
The isSuffixOf function takes two lists and returns True iff the first list is a suffix of the second. The second list must be finite.
\begin{lstlisting}[frame=single]
isSuffixOf::(Eq a) => [a] -> [a] -> Bool
isSuffixOf x y =  reverse x `isPrefixOf` reverse y
\end{lstlisting}

\funcSignature{isInfixOf::Eq\; a\Rightarrow [a]\rightarrow [a]\rightarrow Bool}
The isInfixOf function takes two lists and returns True iff the first list is contained, wholly and intact, anywhere within the second.
\begin{align*}
	&\text{isInfixOf \enquote{Haskell} \enquote{I really like Haskell.} == True}\\
	&\text{isInfixOf \enquote{Ial} \enquote{I really like Haskell.} == False}
\end{align*}
\begin{lstlisting}[frame=single]
isInfixOf::(Eq a) => [a] -> [a] -> Bool
isInfixOf needle haystack = any (isPrefixOf needle) (tails haystack)
\end{lstlisting}

\funcSignature{isSubsequenceOf::Eq\; a\Rightarrow [a]\rightarrow [a]\rightarrow Bool}
The isSubsequenceOf function takes two lists and returns True if all the elements of the first list occur, in order, in the second. The elements do not have to occur consecutively.\\
isSubsequenceOf x y is equivalent to elem x (subsequences y).
\skipped