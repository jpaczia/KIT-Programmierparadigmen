\section{Indexing Lists}
These functions treat a list xs as a indexed collection, with indices ranging from 0 to length xs - 1.

\funcSignature{(!!) :: [a] \rightarrow Int \rightarrow a}
List index (subscript) operator, starting from 0 (first element has index 0). It is an instance of the more general genericIndex, which takes an index of any integral type.

\funcSignature{elemIndex :: Eq\; a \Rightarrow a \rightarrow [a] \rightarrow Maybe Int}
The elemIndex function returns the index of the first element in the given list which is equal (by ==) to the query element, or Nothing if there is no such element.
\begin{lstlisting}[frame=single]
elemIndex       :: Eq a => a -> [a] -> Maybe Int
elemIndex x     = findIndex (x==)
\end{lstlisting}

\funcSignature{elemIndices :: Eq\; a \Rightarrow a \rightarrow [a] \rightarrow [Int] }
The elemIndices function extends elemIndex, by returning the indices of all elements equal to the query element, in ascending order.
\begin{lstlisting}[frame=single]
elemIndices     :: Eq a => a -> [a] -> [Int]
elemIndices x   = findIndices (x==)
\end{lstlisting}

\funcSignature{findIndex :: (a \rightarrow Bool) \rightarrow [a] \rightarrow Maybe Int}
The findIndex function takes a predicate and a list and returns the index of the first element in the list satisfying the predicate, or Nothing if there is no such element.
\begin{lstlisting}[frame=single]
findIndex       :: (a -> Bool) -> [a] -> Maybe Int
findIndex p     = listToMaybe . findIndices p
\end{lstlisting}

\funcSignature{findIndices :: (a \rightarrow Bool) \rightarrow [a] \rightarrow [Int] }
The findIndices function extends findIndex, by returning the indices of all elements satisfying the predicate, in ascending order.
\skipped