\section{List transformations}

\funcSignature{map :: (a \rightarrow b) \rightarrow [a] \rightarrow [b] }
map f xs is the list obtained by applying f to each element of xs, i.e.,
\begin{align*}
	&map\; f\; [x1, x2, ..., xn] == [f\; x1, f\; x2, ..., f\; xn]\\
	&map\; f\; [x1, x2, ...] == [f\; x1, f\; x2, ...]
\end{align*}

\funcSignature{reverse :: [a] \rightarrow [a] }
reverse xs returns the elements of xs in reverse order. xs must be finite.

\funcSignature{intersperse :: a \rightarrow [a] \rightarrow [a] }
The intersperse function takes an element and a list and `intersperses' that element between the elements of the list. For example,
\codeEx{intersperse ',' \enquote{abcde} == \enquote{a,b,c,d,e}}

\begin{lstlisting}[frame=single]
intersperse             :: a -> [a] -> [a]
intersperse _   []      = []
intersperse _   [x]     = [x]
intersperse sep (x:xs)  = x : sep : intersperse sep xs
\end{lstlisting}

\funcSignature{intercalate :: [a] \rightarrow [[a]] \rightarrow [a] }
intercalate xs xss is equivalent to (concat (intersperse xs xss)). It inserts the list xs in between the lists in xss and concatenates the result.

\begin{lstlisting}[frame=single]
intercalate :: [a] -> [[a]] -> [a]
intercalate xs xss = concat (intersperse xs xss)
\end{lstlisting}

\funcSignature{transpose :: [[a]] \rightarrow [[a]] }
The transpose function transposes the rows and columns of its argument. For example,
\codeEx{transpose [[1,2,3],[4,5,6]] == [[1,4],[2,5],[3,6]]}
If some of the rows are shorter than the following rows, their elements are skipped:
\codeEx{transpose [[10,11],[20],[],[30,31,32]] == [[10,20,30],[11,31],[32]]}

\begin{lstlisting}[frame=single]
intercalate :: [a] -> [[a]] -> [a]
intercalate xs xss = concat (intersperse xs xss)
\end{lstlisting}

\funcSignature{subsequences :: [a] \rightarrow [[a]] }
The subsequences function returns the list of all subsequences of the argument.
\codeEx{subsequences \enquote{abc} == [\enquote{}, \enquote{a},\enquote{b},\enquote{ab},\enquote{c},\enquote{ac},\enquote{bc},\enquote{abc}]}
\begin{lstlisting}[frame=single]
transpose               :: [[a]] -> [[a]]
transpose []             = []
transpose ([]   : xss)   = transpose xss
transpose ((x:xs):xss) = (x:[h | (h:_) <- xss]) : transpose (xs:[t | (_:t) <- xss])
\end{lstlisting}

\funcSignature{permutations :: [a] \rightarrow [[a]] }
The permutations function returns the list of all permutations of the argument.
\codeEx{permutations \enquote{abc} == [\enquote{abc},\enquote{bac},\enquote{cba},\enquote{bca},\enquote{cab},\enquote{acb}]}
\begin{lstlisting}[frame=single]
permutations:: [a] -> [[a]]
permutations xs0 =  xs0 : perms xs0 []
  where
    perms [] _  = []
    perms (t:ts) is = foldr interleave (perms ts (t:is)) (permutations is)
      where 
        interleave xs r = let (_,zs) = interleave1 id xs r in zs
        interleave1 _ [] r = (ts, r)
        interleave1 f (y:ys) r = let (us,zs) = interleave1 (f . (y:)) ys r
                                 in  (y:us, f (t:y:us) : zs)
\end{lstlisting}