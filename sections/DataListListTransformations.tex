\section{List transformations}

\funcSignature{map :: (a \rightarrow b) \rightarrow [a] \rightarrow [b] }
map f xs is the list obtained by applying f to each element of xs, i.e.,
\begin{align*}
	&map f [x1, x2, ..., xn] == [f x1, f x2, ..., f xn]\\
	&map f [x1, x2, ...] == [f x1, f x2, ...]
\end{align*}

\funcSignature{reverse :: [a] \rightarrow [a] }
reverse xs returns the elements of xs in reverse order. xs must be finite.

\funcSignature{intersperse :: a \rightarrow [a] \rightarrow [a] }
The intersperse function takes an element and a list and `intersperses' that element between the elements of the list. For example,
\codeEx{intersperse ',' \enquote{abcde} == \enquote{a,b,c,d,e}}

\funcSignature{intercalate :: [a] \rightarrow [[a]] \rightarrow [a] }
intercalate xs xss is equivalent to (concat (intersperse xs xss)). It inserts the list xs in between the lists in xss and concatenates the result.

\funcSignature{transpose :: [[a]] \rightarrow [[a]] }
The transpose function transposes the rows and columns of its argument. For example,
\codeEx{transpose [[1,2,3],[4,5,6]] == [[1,4],[2,5],[3,6]]}
If some of the rows are shorter than the following rows, their elements are skipped:
\codeEx{transpose [[10,11],[20],[],[30,31,32]] == [[10,20,30],[11,31],[32]]}

\funcSignature{subsequences :: [a] \rightarrow [[a]] }
The subsequences function returns the list of all subsequences of the argument.
\codeEx{subsequences \enquote{abc} == [\enquote{}, \enquote{a},\enquote{b},\enquote{ab},\enquote{c},\enquote{ac},\enquote{bc},\enquote{abc}]}

\funcSignature{permutations :: [a] \rightarrow [[a]] }
The permutations function returns the list of all permutations of the argument.
\codeEx{permutations \enquote{abc} == [\enquote{abc},\enquote{bac},\enquote{cba},\enquote{bca},\enquote{cab},\enquote{acb}]}