\chapter{Compiler}
\section{Java-Bytecode \href{https://docs.oracle.com/javase/specs/jvms/se10/jvms10.pdf}{The Java Virtual Machine Specification}}
\subsection{Präfixe / Suffixe}
\begin{table}[h]
	\centering
	\label{my-label}
	\begin{tabular}{l|l}
		Präfix / Suffix	& Operand Typ\\ \hline
		i	&	integer		\\ \hline
		l	&	long		\\ \hline
		s	&	short		\\ \hline
		b	&	byte		\\ \hline
		c	&	character	\\ \hline
		f	&	float		\\ \hline
		d	&	double		\\ \hline
		a	&	reference	\\
	\end{tabular}
\end{table}
\subsection{Lesen / Schreiben von lokalen Variablen}
\begin{table}[h]
	\centering
	\label{my-label}
	\begin{tabular}{l|l|l|l}
		Befehl & Parameter & Beschreibung & Beispiel \\ \hline
		iconst\_x	&	\multlineTable{$x\in \{0,1,2,3,4,5,m1\}$\\ } & \multlineTable{lädt die int-Konstante x\\ m1 steht für Konstante \enquote{-1}} & \multlineTable{\texttt{iconst\_1}\\ lädt den int-Wert \enquote{1} auf den Stack} \\ \hline
		\textbf{TYPE}load\_x	& 	\multlineTable{x: Index der lokalen\\ Variable $x\in \{ 0,1,2,3 \}$\\ (x gehört zum Befehl,\\ es gibt pro x ein Befehl;\\ d.h. x ist keine Variable) }	&	\multlineTable{lädt den Wert der Variable mit Typ\\ \enquote{TYPE} mit Index x auf den Stack} 	& \multlineTable{\texttt{iload\_2}\\ lade Wert von der Variable mit\\ Index 2 und dem Typ \enquote{Integer}}      \\ \hline
		
		\textbf{TYPE}store\_x	&	\multlineTable{x: Index der Variable\\ $x\in \{ 0,1,2,3 \}$\\ (x gehört zum Befehl,\\ es gibt pro x ein Befehl;\\ d.h. x ist keine Variable) }	&	\multlineTable{speichert den obersten Wert auf dem\\ Stack mit Typ \enquote{TYPE} in Variable\\ mit Index x}	& \multlineTable{\texttt{istore\_2}\\ speichere den obersten Wert auf\\ dem Stack vom Typ \enquote{Integer}\\ in Variable 2 }	\\ \hline
		
		\textbf{TYPE}load x	& \multlineTable{x: Index der lokalen\\ Variable $0 \leq x \leq 255$\\ (x mit 1 Byte darstellbar) }	&	\multlineTable{lädt den Wert der Variable mit Typ\\ \enquote{TYPE} mit Index x auf den Stack} 	& \multlineTable{\texttt{iload 7}\\ lade Wert von der Variable mit\\ Index 7 und dem Typ \enquote{Integer}}      \\ \hline
		
		\textbf{TYPE}store x	& \multlineTable{x: Index der lokalen\\ Variable $0 \leq x \leq 255$\\ (x mit 1 Byte darstellbar) }	&	\multlineTable{speichert den obersten Wert auf dem\\ Stack mit Typ \enquote{TYPE} in Variable\\ mit Index x}	& \multlineTable{\texttt{istore 7}\\ speichere den obersten Wert auf\\ dem Stack vom Typ \enquote{Integer}\\ in Variable 7}	\\ \hline	
		
		bipush & \open & Beschreibung & Beispiel \\ \hline	
		
		ldc & \open  & Beschreibung & Beispiel \\ \hline
	\end{tabular}
\end{table}
\myparagraph{Vergleich \enquote{iload\_x} vs. \enquote{iload x}}
\begin{itemize}
	\item \enquote{iload\_x} (mit Unterstrich): $x \in \{ 0,1,2,3 \}$ ist ein (einziger) Befehl, ohne Parameter. \enquote{x} ist schon im Opcode enthalten. Der Befehl besteht aus 1 Byte.
	\item \enquote{iload x} (ohne Unterstrich): $0 \leq x \leq 255$ ist ein Befehl, mit Parameter \enquote{x}. x ist nicht im Opcode enthalten. \enquote{iload x} funktioniert mit allen Zahlen x, die in ein Byte passen.
\end{itemize}
Da die Befehle mit Unterstrich Platz sparen, werden sie von realen Compilern bevorzugt; vorausgesetzt x ist klein genug.

\subsection{Lesen / Schreiben von Feldern}
\begin{table}[h]
	\centering
	\label{my-label}
	\begin{tabular}{l|l|l|l}
		Befehl & Parameter & Beschreibung & Beispiel \\ \hline
		
		putfield & \open & Beschreibung & Beispiel \\ \hline	
		
		getfield & \open & Beschreibung & Beispiel \\ \hline		
		
	\end{tabular}
\end{table}

\subsection{Sprungbefehle}
\begin{table}[h]
	\centering
	\label{my-label}
	\begin{tabular}{l|l|l|l|}
		Befehl & Parameter  & Beschreibung & Beispiel \\ \hline		
		ifle LABEL & \multlineTable{LABEL: Label, zu dem\\ gesprungen werden soll,\\ falls die Bedingung erfüllt ist}   & \multlineTable{Wenn der oberste Wert auf dem\\ Stack \textbf{kleiner oder gleich 0} ist,\\ dann springe zu dem geg. Label} & \multlineTable{ifle then\\ springe zu Label \enquote{then},\\ 
			\slide{73}{411} } \\ \hline		
			
		ifge LABEL & \multlineTable{LABEL: analog zu \enquote{ifle} }   & \multlineTable{Wenn der oberste Wert auf dem\\ Stack \textbf{größer oder gleich 0} ist,\\ dann springe zu dem geg. Label} & \multlineTable{ifge then\\ analog zu \enquote{ifle} } \\ \hline
		
		ifeq LABEL & \multlineTable{LABEL: analog zu \enquote{ifle}}   & \multlineTable{Wenn der oberste Wert\\ auf dem Stack \textbf{gleich 0} ist,\\ dann springe zu dem geg. Label} & \multlineTable{ifeq then\\  analog zu \enquote{ifle} } \\ \hline
		
		ifne LABEL & \multlineTable{LABEL: analog zu \enquote{ifle}}   & \multlineTable{Wenn der oberste Wert\\ auf dem Stack \textbf{nicht gleich 0} ist,\\ dann springe zu dem geg. Label} & \multlineTable{ifne then\\  analog zu \enquote{ifle} } \\ \hline	
		
		ifgt LABEL & \multlineTable{LABEL: analog zu \enquote{ifle}}   & \multlineTable{Wenn der oberste Wert\\ auf dem Stack \textbf{größer 0} ist,\\ dann springe zu dem geg. Label} & \multlineTable{ifgt then\\  analog zu \enquote{ifle} } \\ \hline		
		
		iflt LABEL & \multlineTable{LABEL: analog zu \enquote{ifle}}   & \multlineTable{Wenn der oberste Wert\\ auf dem Stack \textbf{kleiner 0} ist,\\ dann springe zu dem geg. Label} & \multlineTable{iflt then\\  analog zu \enquote{ifle} } \\ \hline	
		
		ifnull LABEL & \multlineTable{LABEL: analog zu \enquote{ifle}}  & \multlineTable{Wenn der oberste Wert (/Referenz)\\ auf dem Stack \textbf{\texttt{NULL} ist},\\ dann springe zu dem geg. Label} & \multlineTable{ifnull then\\  analog zu \enquote{ifle}} \\ \hline	
			
		ifnonnull LABEL & \multlineTable{LABEL: analog zu \enquote{ifle}}  & \multlineTable{Wenn der oberste Wert (/Referenz)\\ auf dem Stack \textbf{nicht \texttt{NULL} ist},\\ dann springe zu dem geg. Label} & \multlineTable{ifnonnull then\\  analog zu \enquote{ifle} }\\ \hline
		
		if\_icmpeq LABEL & \multlineTable{LABEL: analog zu \enquote{ifle}}  & \multlineTable{Wenn die beiden obersten Integer-\\Werte auf dem Stack \textbf{gleich} sind,\\ springe zu dem geg. Label} & \multlineTable{if\_icmpeq then\\ springe zu then\\ \slide{73}{429} } \\ \hline
		
		if\_icmpne LABEL & \multlineTable{LABEL: analog zu \enquote{ifle}}  & \multlineTable{Wenn die beiden obersten Integer-\\Werte auf dem Stack \textbf{nicht gleich}\\ sind, springe zu dem geg. Label} &  \multlineTable{if\_icmpne then\\ analog zu \enquote{if\_icmpeq} } \\ \hline
		
		if\_icmpge LABEL 
		& \multlineTable{LABEL: analog zu \enquote{ifle}}  
		& \multlineTable{Wenn der oberste Integer-Wert auf\\ dem Stack \textbf{gleich oder größer} als\\ der zweite ist, springe zu dem geg. Label} 
		&  \multlineTable{if\_icmpge then\\ analog zu \enquote{if\_icmpeq}} \\ \hline
		
		if\_icmpgt LABEL 
		& \multlineTable{LABEL: analog zu \enquote{ifle}}  
		& \multlineTable{Wenn der oberste Integer-Wert\\ auf dem Stack \textbf{größer} als der\\ zweite ist, springe zu dem geg. Label} 
		&  \multlineTable{if\_icmpgt then\\ analog zu \enquote{if\_icmpeq}} \\ \hline
		
		if\_icmple LABEL & \multlineTable{LABEL: analog zu \enquote{ifle}}  
		& \multlineTable{Wenn der oberste Integer-Wert auf\\ dem Stack \textbf{kleiner oder gleich} als\\ der zweite ist, springe zu dem geg. Label} 
		&  \multlineTable{if\_icmple then\\ analog zu \enquote{if\_icmpeq}} \\ \hline
		
		if\_icmplt LABEL & \multlineTable{LABEL: analog zu \enquote{ifle}}  & \multlineTable{Wenn der oberste Integer-Wert auf\\ dem Stack \textbf{kleiner} als der zweite ist,\\ springe zu dem geg. Label} &  \multlineTable{if\_icmplt then\\ analog zu \enquote{if\_icmpeq}} \\ \hline
		
		goto LABEL	& \multlineTable{LABEL: analog zu \enquote{ifle}}  & \multlineTable{Springe bedingungslos zu dem geg.\\ Label} & \multlineTable{goto done\\ springe zu Label \enquote{done}\\ \slide{73}{411} }\\ \hline		
	\end{tabular}
\end{table}
\subsection{Methodenaufrufe}
\begin{table}[h]
	\centering
	\label{my-label}
	\caption{\href{https://www.javaworld.com/article/2860079/learn-java/invokedynamic-101.html}{Quelle}}
	\begin{tabular}{l|l|l|l|}
		Befehl & Parameter & Beschreibung & Beispiel \\ \hline
		
		invokevirtual \#INDEX 
		& \multlineTable{INDEX: Index der\\ aufzurufenden Methode} 
		& \multlineTable{Rufe die \textbf{nicht statische, public}\\ \textbf{oder protected} Methode mit dem\\ geg. Index auf. Die Parameter der\\ Methode werden automatisch in den\\ Operandenstack geladen, \textbf{beginnend}\\ \textbf{bei Variablen-Index 1}\\ (Variablen-Index 0: this-Variable / Referenz)} 
		& \multlineTable{invokevirtual \#2\\  Rufe die Methode\\ mit Index 2 auf\\ \slide{73}{419} } \\ \hline
		
		invokestatic \#INDEX 
		& \multlineTable{INDEX:  analog zu\\ \enquote{invokevirtual}} 
		& \multlineTable{Rufe die \textbf{statische} Methode mit dem geg.\\ Index auf. Die Parameter der Methode\\ werden automatisch in den Operandenstack\\ geladen, \textbf{beginnend bei Variablen-Index 0}\\ (statische Methode, somit keine \enquote{this}-Variable)} 
		& \multlineTable{invokestatic \#2\\ analog zu \enquote{invokevirtual}}\\  \hline
		
		invokespecial \#INDEX 
		& \multlineTable{INDEX:  analog zu\\ \enquote{invokevirtual}} 
		& \multlineTable{Rufe die \textbf{private, Superklassen- oder Konstruktor-} Methode mit dem geg. Index auf. Die Parameter der\\ Methode werden automatisch in den\\ Operandenstack geladen, \textbf{beginnend}\\ \textbf{bei Variablen-Index 1}\\ (Variablen-Index 0: this-Variable / Referenz)} 
		& \multlineTable{invokespecial \#2 \\ analog zu \enquote{invokevirtual}} \\ \hline
		
		invokedynamic \#INDEX 
		& \multlineTable{INDEX:  analog zu\\ \enquote{invokevirtual}} 
		& Beschreibung 
		& \multlineTable{invokedynamic \#2 \\ analog zu \enquote{invokevirtual}} \\ \hline
		
		invokeinterface \#INDEX 
		& \multlineTable{INDEX:  analog zu\\ \enquote{invokevirtual}} 
		& Beschreibung 
		& \multlineTable{invokeinterface \#2 \\ analog zu \enquote{invokevirtual}} \\ \hline
		
		
	\end{tabular}
\end{table}


\subsection{Objekterzeugung}
\begin{table}[h]
	\centering
	\label{my-label}
	\begin{tabular}{l|l|l|l}
		Befehl & Parameter & Beschreibung & Beispiel \\ \hline
		
		newarray & \open & Beschreibung & Beispiel \\ \hline	
		
	\end{tabular}
\end{table}

\subsection{Arithmetische Berechnungen}
\begin{table}[h]
	\centering
	\label{my-label}
	\begin{tabular}{l|l|l|l}
		Befehl & Parameter & Beschreibung & Beispiel \\ \hline
		
		\textbf{TYPE}mul & - & \multlineTable{multipliziert zwei Werte vom Typ\\ \enquote{TYPE} und lädt das Ergebnis als\\ obersten Wert auf den Stack} & \multlineTable{\texttt{imul}} \\ \hline
		
		\textbf{TYPE}div & - & \multlineTable{dividiert zwei Werte vom Typ\\ \enquote{TYPE} und lädt das Ergebnis als\\ obersten Wert auf den Stack} & \multlineTable{\texttt{idiv}} \\ \hline
		
		\textbf{TYPE}add & - & \multlineTable{addiert zwei Werte vom Typ\\ \enquote{TYPE} und lädt das Ergebnis als\\ obersten Wert auf den Stack} & \multlineTable{\texttt{iadd}} \\ \hline
		
		\textbf{TYPE}sub & - & \multlineTable{subtrahiert zwei Werte vom Typ\\ \enquote{TYPE} und lädt das Ergebnis als\\ obersten Wert auf den Stack} & \multlineTable{\texttt{iadd}} \\ \hline
		
		\textbf{TYPE}neg & - & \multlineTable{negiert einen Wert vom Typ\\ \enquote{TYPE} und lädt das Ergebnis als\\ obersten Wert auf den Stack} & \multlineTable{\texttt{ineg}} \\ \hline
		
		
		iinc index, const & \multlineTable{index:\\ Index der zu\\ inkrementierenden Variable\\ const:\\ Wert um den die Variable\\ inkrementiert werden soll}  & \multlineTable{Inkrementiere die Variable mit dem\\ geg. Index um den geg. konstanten\\ Wert} & \multlineTable{iinc 1, 1\\ inkrementiere die Variable\\ mit Index 1 um 1\\ iinc 1, -1\\ inkrementiere die Variable\\ mit Index 1 um (-1) } \\ \hline
		
	\end{tabular}
\end{table}