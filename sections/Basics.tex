\section{Haskell}
\begin{itemize}
	\item equal:: \enquote{==}
	\item not equal:: \enquote{/=}
\end{itemize}

\funcSignature{changeListAt::(a -> a) -> Int -> [a] -> [a]} 
\exRef{SS 16, Nr. 1}\\
\texttt{changeListAt f i list}: Wendet auf das i-te Element (1. Element hat Index 0) der Liste \enquote{list} die Funktion f an.
\begin{lstlisting}
changeListAt f i []=[]
changeListAt f 0 (x:xs) = ((f x):xs)
changeListAt f i (x:xs) = x.(changeListAt f (i-1) xs)
\end{lstlisting}

\funcSignature{isPalindrom::Eq a => [a] -> Bool}
\texttt{isPalindrom list} überprüft es sich bei der Liste \enquote{list} um ein Palindrom handelt 
\begin{lstlisting}
isPalindrom xs = foldl (&&) True (zipWith (\x y -> x==y) xs (myRev1 xs))
\end{lstlisting}

\funcSignature{flatten::[[a]] -> [a]}
Transform a list, possibly holding lists as elements into a `flat' list by replacing each list with its elements (recursively). 
\begin{lstlisting}
data NestedList a = Elem a | List [NestedList a]

flatten (Elem a)      = [a]
flatten (List [])     = []
flatten (List (x:xs)) = (flatten x) ++ (flatten (List xs))
flatten (List [])     = []
\end{lstlisting}

\funcSignature{compress::Eq a => [a] -> [a]}
Entferne aufeinanderfolgende Duplikate in einer Liste
\begin{lstlisting}
compress []     = []
compress (x:xs) = x : (compress $ dropWhile (== x) xs)
\end{lstlisting}

\funcSignature{pack::Eq a => [a] -> [[a]]}
Speichere aufeinanderfolgende Duplikate aus der Eingabeliste in Unterlisten. Falls sich Elemente aus der Eingabeliste wiederholen und nicht direkt aufeinanderfolgen, werden diese in separate Unterlisten gespeichert.
\begin{lstlisting}
pack::Eq a => [a] -> [[a]]
pack [] = []
pack (x:xs) = (x:(filter (==x) xs)):(pack $ filter (/=x) xs)
\end{lstlisting}

\funcSignature{duplicate::[a] -> [a]}
Dupliziere die Elemente einer Liste
\begin{lstlisting}
duplicate []     = []
duplicate (x:xs) = x x:(duplicate xs)
\end{lstlisting}

\funcSignature{replicate::[a] -> Int -> [a]}
\texttt{replicate list n} repliziert die Elemente der Liste \enquote{list} n-mal
\begin{lstlisting}
repli xs n = concat [f x | x <- xs]
  where f x = take n (repeat x)
\end{lstlisting}

\funcSignature{dropEvery::[a] -> Int -> [a]}
\texttt{dropEvery list n} entfernt jedes n-te Element aus der Liste
$$\text{dropEvery [1,2,3,4] 2 == [1,3]}$$
\begin{lstlisting}
dropEvery list count = helper list count count
  where 
    helper [] _ _ = []
    helper (x:xs) count 1 = helper xs count count
    helper (x:xs) count n = x : (helper xs count (n - 1))
\end{lstlisting}

\funcSignature{rotate::[a] -> Int -> [a]}
\texttt{rotate list n} verschiebe die Elemente der Liste um n Stellen nach links
$$\text{rotate \enquote{abcdefgh} 3 == \enquote{defghabc}}$$
$$\text{rotate \enquote{abcdefgh} (-2) == \enquote{ghabcdef}}$$
\begin{lstlisting}
rotate xs n = drop nn xs ++ take nn xs
  where nn = n `mod` length xs
\end{lstlisting}