\chapter{Prolog}
\section{Vergleich arithmetischer Ausdrücke}
\begin{compactitem}
	\item gleich: \enquote{=:=}
	\item ungleich: \enquote{=\textbackslash =}
	\item kleiner: \enquote{$<$}
	\item kleiner-gleich : \enquote{$=<$}
	\item größer: \enquote{$>$}
	\item größer-gleich: \enquote{$>=$}
\end{compactitem}

\section{Funktionen für Listen}
\begin{compactitem}
	\item member \slide{31}{243}: Überprüfe ob Element in Liste enthalten
		\begin{lstlisting}
		member(X,[X|T]).
		member(X,[Y|T]):-member(X, T).
		\end{lstlisting}
	\item append \slide{31}{243}: Hänge eine Liste an eine andere
		\begin{lstlisting}
		append([],L,L).
		append([X|R],L,[X|T]):-append(R,L,T).
		\end{lstlisting}
		\enquote{Die Konkatenation von [] und L ist L. Wenn die Konkatenation von R und L die Liste T ergibt, dann ergibt die Konkatenation von [X|R] und L die Liste [X|T].}
	\item reverse \slide{31}{245}: 		
		\begin{lstlisting}
		reverse([],[]).
		reverse([X|R],Y):- reverse(R,Y1), append(Y1,[X],Y).
		\end{lstlisting}
		effizienter:
		\begin{lstlisting}
		reverse(X,Y):-reverse(X, [], Y).
		reverse([],Y,Y).
		reverse([X|R],A,Y):-reverse(R,[X|A],Y).
		\end{lstlisting}
	\item Quicksort \slide{31}{247}:
		\begin{lstlisting}
		qsort([],[]).
		qsort([X|R],Y):- split(X,R,R1,R2),
				 qsort(R1,Y1),
				 qsort(R2,Y2),
				 append(Y1,[X|Y2],Y).
		split(X,[],[],[]).
		split(X,[H|T],[H|R],Y):- X>H, split(X,T,R,Y).
		split(X,[H|T],R,[H|Y]):- X=<H, split(X,T,R,Y).
		\end{lstlisting}
	\item Listenpermutation \slide{31}{248}:
		\begin{lstlisting}
		permute([],[]).
		permute([X|R],P):- permute(R,P1),append(A,B,P1),append(A,[X|B],P).
		\end{lstlisting}
	\item lösche alle Elemente X aus Liste \exRef{Üb 8, Nr. 3}:
		\begin{lstlisting}
		del([],_,[]).
		del([X|T1],X,L2)    :- del(T1,X,L2).
		del([Y|T1],X,[Y|T2]):- del(T1,X,T2), not(X=Y).
		\end{lstlisting}	
	\item Listenlänge \exRef{WS 12/13, Nr. 3a}
		\begin{lstlisting}
		length([],0).
		length([_|R],NewLength):- length(R,Length), NewLength is Length +1.
		\end{lstlisting}	
	\item alle möglichen Zerlegungen (Anfang- und Endteil) einer Liste \exRef{SS 13, Nr. 4b}
		\begin{lstlisting}
		splits(L,([],L)).
		splits([X|L],([X|S], E)):- splits(L,(S,E)).
		\end{lstlisting}	
	\item alle Teillisten einer Liste\exRef{WS 15/16, Nr. 3a}
		\begin{lstlisting}
		sublists([],[]).
		sublists([X|L],[X|L2]):-sublists(L,L2).
		sublists([_|L],L2):-sublists(L,L2).
		\end{lstlisting}
	\item Test auf Duplikate:
		\begin{lstlisting}
		noDuplicates([]).
		noDuplicates([H|T]):-not(member(H,T)),noDuplicates(T).
		\end{lstlisting}
	\item Entferne aufeinanderfolgende Duplikate:
		\begin{lstlisting}
		removeDuplicates([],[]).
		removeDuplicates([H|[H|T]],L):- removeDuplicates([H|T],L).
		removeDuplicates([H|T],[H|L]):- removeDuplicates(T,L).
		\end{lstlisting}
	\item Entferne alle Duplikate (auch wenn doppelte Elemente nicht direkt hintereinander sind):
		\begin{lstlisting}
		removeAllDuplicates([],[]).
		removeDuplicates([H|T],[H|L]):- deleteElem(T,H,L1), removeDuplicates(L1,L).
		\end{lstlisting}
\end{compactitem}

\section{Sonstige}
\begin{compactitem}
	\item atom(Term): True, falls Term mit einem Atom instanziiert ist \slide{32}{272}
	\item atomic(Term): True, falls Term mit einem Atom instanziiert ist \slide{32}{272}
	\item integer(Term): True, falls Term mit einem Integer instanziiert ist
	\item var(Term): True, falls Term aktuell eine freie Variable ist
\end{compactitem}