\section{Building lists}
\subsection{Scans}
\funcSignature{scanl::(b \rightarrow a \rightarrow b) \rightarrow b \rightarrow [a] \rightarrow [b] }
scanl is similar to foldl, but returns a list of successive reduced values from the left:
\codeEx{scanl f z [x1, x2, ...] == [z, z `f` x1, (z `f` x1) `f` x2, ...]}
Note that
\codeEx{last (scanl f z xs) == foldl f z xs.}

\funcSignature{scanl'::(b \rightarrow a \rightarrow b) \rightarrow b \rightarrow [a] \rightarrow [b] }
A strictly accumulating version of scanl

\funcSignature{scanl1::(a \rightarrow a \rightarrow a) \rightarrow [a] \rightarrow [a] }
scanl1 is a variant of scanl that has no starting value argument:
\codeEx{scanl1 f [x1, x2, ...] == [x1, x1 `f` x2, ...]}

\funcSignature{scanr::(a \rightarrow b \rightarrow b) \rightarrow b \rightarrow [a] \rightarrow [b] }
scanr is the right-to-left dual of scanl. Note that
\codeEx{head (scanr f z xs) == foldr f z xs.}

\funcSignature{scanr1::(a \rightarrow a \rightarrow a) \rightarrow [a] \rightarrow [a] }
scanr1 is a variant of scanr that has no starting value argument.

\subsection{Accumulating maps}
\funcSignature{mapAccumL::Traversable\; t \Rightarrow (a \rightarrow b \rightarrow (a, c)) \rightarrow a \rightarrow t b \rightarrow (a, t\; c) }
The mapAccumL function behaves like a combination of fmap and foldl; it applies a function to each element of a structure, passing an accumulating parameter from left to right, and returning a final value of this accumulator together with the new structure.

\funcSignature{mapAccumR::Traversable\; t \Rightarrow (a \rightarrow b \rightarrow (a, c)) \rightarrow a \rightarrow t b \rightarrow (a, t\; c) }
The mapAccumR function behaves like a combination of fmap and foldr; it applies a function to each element of a structure, passing an accumulating parameter from right to left, and returning a final value of this accumulator together with the new structure.

\subsection{Infinite lists}
\funcSignature{iterate::(a \rightarrow a) \rightarrow a \rightarrow [a] }
iterate f x returns an infinite list of repeated applications of f to x:
\codeEx{iterate f x == [x, f x, f (f x), ...]}

\funcSignature{repeat::a \rightarrow [a] }
repeat x is an infinite list, with x the value of every element.

\funcSignature{replicate::Int \rightarrow a \rightarrow [a] }
replicate n x is a list of length n with x the value of every element. It is an instance of the more general genericReplicate, in which n may be of any integral type.

\funcSignature{cycle::[a] \rightarrow [a] }
cycle ties a finite list into a circular one, or equivalently, the infinite repetition of the original list. It is the identity on infinite lists.
\example
\begin{lstlisting}
Input: take 10 (cycle [1,2,3])
Output: [1,2,3,1,2,3,1,2,3,1]
\end{lstlisting}

\subsection{Unfolding}
\funcSignature{unfoldr::(b \rightarrow Maybe\; (a, b)) \rightarrow b \rightarrow [a] }
The unfoldr function is a `dual' to foldr: while foldr reduces a list to a summary value, unfoldr builds a list from a seed value. The function takes the element and returns Nothing if it is done producing the list or returns Just (a,b), in which case, a is a prepended to the list and b is used as the next element in a recursive call. For example,
\codeEx{iterate f == unfoldr (\textbackslash x -\textgreater Just (x, f x))}
In some cases, unfoldr can undo a foldr operation:
\codeEx{unfoldr f' (foldr f z xs) == xs}
if the following holds:
\begin{align*}
	&f'\; (f\; x y) = Just\; (x,y)\\
	&f'\; z       = Nothing
\end{align*}
A simple use of unfoldr:
\begin{align*}
	&unfoldr\; (\backslash b \rightarrow if\; b == 0\; then\; Nothing\; else\; Just\; (b, b-1))\; 10\\
	&[10,9,8,7,6,5,4,3,2,1]
\end{align*}