\chapter{Design by Contract}
\section{JML}
\subsection{Basic Syntax}
\begin{table}[h]
	\centering
	\label{my-label}
	\begin{tabular}{l|l}
		Syntax    & Bedeutung      															\\ \hline
		$a ==> b$ & a impliziert b 															\\ \hline
		$a <==>$  & a und b äquivalent            											\\ \hline
		$a <=!=>$  & a und b \textbf{nicht} äquivalent ($a \nleftrightarrow b$)            	\\ \hline
		\textbackslash result  & Ergebnis der Methode            							\\ \hline
		\textbackslash old(E)  & Wert von E, bevor die Methode ausgeführt wurde        		\\ \hline	
		(\textbackslash forall declaration; range-expression; body-expression)  
		& \multlineTable{(\textbackslash forall int i; 0 \textless = i \&\& i \textless size; \textbackslash old(elements[i])==elements[i])\\ für alle i zwischen 0 und \enquote{size} gilt: \textbackslash old(elements[i])==elements[i])}\\ \hline	
		(\textbackslash exists declaration; range-expression; body-expression)  
		& \multlineTable{(\textbackslash exists int i; 0 \textless = i \&\& i \textless size; \textbackslash old(elements[i])==elements[i])\\ es gibt ein i zwischen 0 und \enquote{size}, für das gilt:\\ \textbackslash old(elements[i])==elements[i])}\\ \hline
	\end{tabular}
\end{table}

\lstset{language=Java}

\begin{lstlisting}
/*@ requires size > 0;
@ ensures size == \old(size) - 1
@ ensures \result == \old(top())
@ ensures (\forall int i; 0 <= i && i < size;
\old(elements[i]) == elements[i]);
@*/
Object pop() { pop logic }

\end{lstlisting}